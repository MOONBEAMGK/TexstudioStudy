\documentclass[]{article}
\usepackage{ctex}
%opening
\title{Hello Texstudio}
\author{MOOBEAMS GK}

\begin{document}

\maketitle

\begin{abstract}
This is an abstract.Semantic segmentation of 3D scenes is one of the
most important tasks in the field of computer vision and has
attracted much attention. In this paper, we propose a novel frame
work for 3D semantic segmentation of aerial photogrammetry
models, which uses orthographic projection to improve efficiency
while still ensuring high precision, and can also be applied to
multiple types of models (i.e., textured mesh or colored point
cloud). In our pipeline, we first obtain RGB images and elevation
images from the 3D scene through orthographic projection, then
use the image semantic segmentation network to segment these
images to obtain pixel-wise semantic predictions, and finally
back-project the segmentation results to the 3D model for
fusion. Specifically, for the image semantic segmentation model,
we design a cross-modality feature aggregation module and a
context guidance module based on category features, which assist
the network in learning more discriminative features between
different objects. For the 2D-3D semantic fusion, we combine
the segmentation results of the 2D images with the geometric
consistency of the 3D models for joint optimization, which further
improves the accuracy of the 3D semantic segmentation. Exten
sive experiments on two large-scale urban scenes demonstrate
the efficiency and feasibility of our algorithm and surpass the
current mainstream 3D deep learning met
\end{abstract}
\newpage
\tableofcontents
\newpage

\section{问题重述}

\subsection{问题的背景}

\subsection{问题的提出}

\section{问题分析}

\subsection{问题一的分析}

\subsection{问题二的分析}

\subsection{问题三的分析}

\section{模型假设}
\section{符号说明}
\section{模型的建立与求解}

\newpage
\subsection{问题一的模型}
\subsubsection{数据的预处理}
\subsubsection{模型的建立}
\subsubsection{模型计算求解}
\subsubsection{问题一的结论}

\newpage
\subsection{问题二的模型}
\subsection{问题三的模型}

	\appendix 
\renewcommand{\appendixname}{Appendix~\Alph{section}}
\section{附表1——XXXXXX}


\end{document}
